%%%%%%%%%%%%%%%%%%%%%%%%%%%%%%%%%%%%%%%
% This is a modified ONE COLUMN version of
% the following template:
% 
% Deedy - One Page Two Column Resume
% LaTeX Template
% Version 1.1 (30/4/2014)
%
% Original author:
% Debarghya Das (http://debarghyadas.com)
%
% Original repository:
% https://github.com/deedydas/Deedy-Resume
%
% IMPORTANT: THIS TEMPLATE NEEDS TO BE COMPILED WITH XeLaTeX
%
% This template uses several fonts not included with Windows/Linux by
% default. If you get compilation errors saying a font is missing, find the line
% on which the font is used and either change it to a font included with your
% operating system or comment the line out to use the default font.
% 
%%%%%%%%%%%%%%%%%%%%%%%%%%%%%%%%%%%%%%
% 
% TODO:
% 1. Integrate biber/bibtex for article citation under publications.
% 2. Figure out a smoother way for the document to flow onto the next page.
% 3. Add styling information for a "Projects/Hacks" section.
% 4. Add location/address information
% 5. Merge OpenFont and MacFonts as a single sty with options.
% 
%%%%%%%%%%%%%%%%%%%%%%%%%%%%%%%%%%%%%%
%
% CHANGELOG:
% v1.1:
% 1. Fixed several compilation bugs with \renewcommand
% 2. Got Open-source fonts (Windows/Linux support)
% 3. Added Last Updated
% 4. Move Title styling into .sty
% 5. Commented .sty file.
%
%%%%%%%%%%%%%%%%%%%%%%%%%%%%%%%%%%%%%%%
%
% Known Issues:
% 1. Overflows onto second page if any column's contents are more than the
% vertical limit
% 2. Hacky space on the first bullet point on the second column.
%
%%%%%%%%%%%%%%%%%%%%%%%%%%%%%%%%%%%%%%

\documentclass[]{deedy-resume-openfont}


\begin{document}

%%%%%%%%%%%%%%%%%%%%%%%%%%%%%%%%%%%%%%
%
%     LAST UPDATED DATE
%
%%%%%%%%%%%%%%%%%%%%%%%%%%%%%%%%%%%%%%
\lastupdated

%%%%%%%%%%%%%%%%%%%%%%%%%%%%%%%%%%%%%%
%
%     TITLE NAME
%
%%%%%%%%%%%%%%%%%%%%%%%%%%%%%%%%%%%%%%



\namesection{Prajjwol}{Dandekhya}
{
\begin{minipage}[t]{.5\linewidth}
%\centering
423 West A Street\\
Apartment 804\\
Moscow Idaho\\
Mobile: 208.874.3391
\end{minipage}
\hfill
\begin{minipage}[t]{.38\linewidth}
%\centering
{ \urlstyle{same}
\raggedleft{} \href{mailto:dprajjwol@gmail.com}{dprajjwol@gmail.com}\\
\raggedleft{} \href{https://github.com/prajjwold}{https://github.com/prajjwold}\\
%\raggedleft{} \href{http://www.pdandekhya.com.np}{http://www.pdandekhya.com.np} \\
\raggedleft{} \href{https://np.linkedin.com/in/pdandekhya}{https://np.linkedin.com/in/pdandekhya}\\
\raggedleft{} \href{https://uidaho.academia.edu/pdandekhya}{https://uidaho.academia.edu/pdandekhya}\\
}
\end{minipage}
}

% \begin{center}
% \huge\color{subheadings}\custombold{LIFELONG LEARNER}
% \end{center}

\section{Education}

%\noindent\rule{\linewidth}{0.4pt}
\runsubsection{Masters in Computer Science}
\descript{| University of Idaho | Moscow, ID}
\location{Expected May 2017| \textbullet{} Cum. GPA: 4.0/4.0}
\sectionsep


\runsubsection{Bachelor in Computer Engineering}
\descript{| Tribhuwan University |
Institute of Engineering | Nepal}
\location{Nov 2008 -Oct 2012 
\textbullet{}Cum. GPA: 3.93 / 4.0}
%\sectionsep
%line separtor
\rule[2.5mm]{\textwidth}{0.4pt}

\section{Experience}
%\noindent\rule{\linewidth}{0.4pt}
\runsubsection{Graduate Research Assistant}
\descript{| University of Idaho }
\location{Jan 2016 – Present | Moscow,ID}
%\vspace{\topsep} % Hacky fix for awkward extra vertical space
\begin{tightemize}
\item Assisted Professor Dr. Axel Krings with his research on Vehicle to Vehicle(V2V) and Vehicle to Infrastructure(V2I) communication.
\item Actively involved in implementation of smart Lighted Raised Pavement Marker in Traffic System.
\item Working in the NS3 simulation of IEEE 802.11p Wave Protocol for vehicular communications.
\end{tightemize}
\sectionsep

%\noindent\rule{\linewidth}{0.4pt}
\runsubsection{Software Engineering Intern}
\descript{| Micron Technology Inc.}
\location{May 2016 – Aug 2016 | Boise,ID}
%\vspace{\topsep} % Hacky fix for awkward extra vertical space
\begin{tightemize}
\item Worked as a developer for in-house software used for process-specification and capital management within Micron.
\item Took the sole responsibility for developing a plugin for Capital Management Module.
\item Used MVVM design pattern and .Net framework.
\end{tightemize}
\sectionsep

% \runsubsection{Teaching Assistant}
% \descript{| University of Idaho }
% \location{Aug 2015 – Dec 2015 | Moscow,ID}
% %\vspace{\topsep} % Hacky fix for awkward extra vertical space
% \begin{tightemize}
% \item Instruct the undergraduate student on Computer Programming Labs (C, C++)
% \item Grading of the lab assignments and tutor the students on their lab assignments.
% \end{tightemize}
% \sectionsep

\runsubsection{Software Engineer}
\descript{| Verisk Information Technologies }
\location{Nov 2012 – Aug 2015 | Kathmandu,Nepal}
%\vspace{\topsep} % Hacky fix for awkward extra vertical space
\begin{tightemize}
\item Developed modules for the US healthcare data analytic product called Medical Intelligence.
\item Worked on Test Automation, Data Verification and Testing Framework Development.
\item Actively Involved in the research oriented project to create a prototype for health-care analytic with the implementation of Hadoop (a growing technology in the field of big-data analytics).
\end{tightemize}
\sectionsep

\runsubsection{Assistant Software Engineer Internship}
\descript{| Deerwalk Services Pvt. Ltd }
\location{May 2012 – Sep 2012 | Kathmandu, Nepal}
\begin{tightemize}
\item Worked on the final academic project related to prototyping of national census data analytic using Hadoop (distributed processing technologies).Hadoop,Hbase,Groovy and Grails
\end{tightemize}
%\sectionsep
%Line separator
\rule[2.5mm]{\textwidth}{0.4pt}

%%%%%%%%%%%%%%%%%%%%%%%%%%%%%%%%%%%%%%
%     RESEARCH
%%%%%%%%%%%%%%%%%%%%%%%%%%%%%%%%%%%%%%

% \section{Research}
% \noindent\rule{\linewidth}{0.4pt}
% \runsubsection{Cornell Robot Learning Lab}
% \descript{| Head Undergrad Research}
% \location{Jan 2014 – Present | Ithaca, NY}
% Worked with \textbf{\href{http://www.cs.cornell.edu/~ashesh/}{Ashesh Jain}} and \textbf{\href{http://www.cs.cornell.edu/~asaxena/}{Prof Ashutosh Saxena}} to create \textbf{PlanIt}, a tool which  learns from large scale user preference feedback to plan robot trajectories in human environments.  Publication submitted.
% \sectionsep

% \runsubsection{Cornell Phonetics Lab}
% \descript{| Head Undergraduate Researcher}
% \location{Mar 2012 – May 2013 | Ithaca, NY}
% Lead the development of \textbf{QuickTongue}, the first ever breakthrough tongue-controlled game with \textbf{\href{http://conf.ling.cornell.edu/~tilsen/}{Prof Sam Tilsen}} to aid in Linguistics research. Publication submitted.
% \sectionsep


% \section{Courses and Certifications}
% %\noindent\rule{\linewidth}{0.4pt}
% \textbullet{}
% \runsubsection{\href{https://www.coursera.org/course/ml}{Machine Learning }}
% \location{ Coursera.org}
% %\sectionsep
% %\runsubsection{\href{https://www.coursera.org/account/accomplishments/records/8vZBCJMMKxLNjqCN}{Programming Mobile Applications for Android Handheld Systems }}
% %\location{Coursera.org | Verified Certificates License 7VP9T9AXHM}
% %\sectionsep
% \textbullet{}
% \runsubsection{\href{https://www.coursera.org/course/webapplications}{Web Application Architectures }}
% \location{Coursera.org}
% %\sectionsep
% \textbullet{}
% \runsubsection{\href{https://www.coursera.org/account/accomplishments/records/V78EeB4s9BkjfqpS}{Pattern-Oriented Software Architectures: Programming Mobile Services for Android Handheld Systems }}
% \location{Coursera.org 
% %| Verified Certificates License A9RTP4HWZT
% }
% %\sectionsep
% \textbullet{}
% \runsubsection{\href{https://courses.edx.org/courses/LinuxFoundationX/LFS101x/2T2014/info}{LFS101x:Introduction to Linux, The Linux Foundation }}
% \location{edx.org %| License 09a57e2b847b4456960ef7a02b6a3e6e
% }
% %\sectionsep
% \textbullet{}
% \runsubsection{\href{https://www.coursera.org/account/accomplishments/records/gbQ597xVkjeYT5Pe}{Programming Cloud Services for Android Handheld Systems }}
% \location{Coursera.org %| Verified Certificates License BF6KFVXJSP
% }
% %\sectionsep
% \textbullet{}
% \runsubsection{\href{https://www.isqi.org/de/certview.html?CertificateID=1}{iSQI\textsuperscript{\textregistered} Certified Agile Tester Certified }}
% \location{iSQI GmbH - International Software Quality %| License 14-CATCert-66156-25
% }
% %\sectionsep
% \textbullet{}
% \runsubsection{\href{https://www.coursera.org/learn/data-structures-optimizing-performance}{Data structures: Measuring and Optimizing Performance }}
% \location{Coursera.org}
% %\sectionsep


% \section{Academic Projects}
% \noindent\rule{\linewidth}{0.4pt}
% \runsubsection{Distributed Approach of analyzing and visualizing census data}
% \location{\\The project focused on the analysis of sample census data using Hadoop, an open source framework for distributed computing. With this project, we demonstrate how the use of distributed computing benefits the efficiency and cost effectiveness in analysis of large scale datasets.}
% \sectionsep
% \runsubsection{Sampaadak (A Nepali Spelling Checking Engine)}
% \location{\\Sampaadak is an application for editing documents typed in Nepali language using Unicode. It provides the feature of auto spelling correction for Nepali words based on Levenshtein distance algorithm.}
% \sectionsep

%%Line separator
%\rule[2.5mm]{\textwidth}{0.4pt}

\section{Languages}
%\noindent\rule{\linewidth}{0.4pt}
\begin{minipage}[t]{.27\linewidth} %
\subsection{Programming}
\location{Over 5000 lines:}
\textbullet{}Java \textbullet{}C \textbullet{} C++ \textbullet{} C\# \textbullet{} PLSQL \textbullet{} Spring Framework  \\ 
\location{Over 1000 lines:}
\textbullet{}  Shell Scripting \textbullet{} PHP \textbullet{} Python \textbullet{} Matlab \\ 
\location{Familiar:}
\textbullet{}Hadoop \textbullet{} HBase \textbullet{} Spark \textbullet{} Android \textbullet{} MySQL \textbullet{}XML \textbullet{} \LaTeX\
%\sectionsep
\end{minipage}
%\hfill
\begin{minipage}[t]{.25\linewidth} 
\subsection{Tools}
\location{Project Management Tools:} \textbullet{}JIRA \textbullet{} Rally\\
\location{Version Control:} \textbullet{}GIT \textbullet{} SVN \\
\location{Others:} \textbullet{}Maven \textbullet{} JUnit \textbullet{} Bamboo\\
\end{minipage}
%\hfill
\begin{minipage}[t]{.23\linewidth} %
\subsection{Graduate Courses}
Evolutionary Computation\\
Machine Learning\\
Computer Networks\\
Applied Security Concepts\\
Data Communications\\
Survivable Systems\\
Fault Tolerance
\end{minipage}
%\hfill
\begin{minipage}[t]{.2\linewidth} %
\subsection{Undergrad Courses}
Data Structures
\\ Software Engineering
\\ Database Concepts
\\ Programming Languages (C/C++)
\\ Digital Signal Processing 
\\ Theory of Computation
\\ Microprocessor
\end{minipage}

% %\hfill
% \begin{minipage}[t]{.2\linewidth} %
% \subsection{Spoken \& Written}
% \location{Native fluency:} English, Nepali\\
% \location{Reading fluency:} English, Nepali\\
% \end{minipage}

%Line separator
\rule[2.5mm]{\textwidth}{0.4pt}

% \section{Courses}
% \begin{minipage}[t]{.5\linewidth} %
% \subsection{Graduate Courses}
% \begin{itemize}
% \item Evolutionary Computation
% \item Machine Learning
% \item Computer Networks
% \item Applied Security Concepts
% \item Data Communications
% \item Survivable Systems
% \item Fault Tolerance
% \end{itemize}

% \end{minipage}
% \begin{minipage}[t]{.4\linewidth} %
% \subsection{Undergrad Courses}
% \begin{itemize}
% \item Data Structures
% \item Software Engineering
% \item Database Concepts
% \item Programming Languages (C/C++)
% \item Digital Signal Processing 
% \item Theory of Computation
% \end{itemize}
% \end{minipage}

% %Line separator
% \rule[2.5mm]{\textwidth}{0.4pt}


\section{AWARDS AND ACHIEVEMENTS}
%\noindent\rule{\linewidth}{0.4pt}
\runsubsection{Winner of Theme Competition on 9th National Technological Festival}
%\descript{|  organized by Locus 2012}
\location{\\2012 | Central Campus Pulchowk, Institute of Engineering, Nepal}
\sectionsep
\runsubsection{First Runner up of Yomari Code Camp 2012}
%\descript{|  organized by Locus 2012}
\location{\\2012 | Central Campus Pulchowk, Institute of Engineering}
\sectionsep
\runsubsection{Winner of Software Competition in KU IT-MEET 2011}
%\descript{|  organized by Kathmandu University Computer Club}
\location{\\2011 | Kathmandu University, Nepal}
%\sectionsep
\end{document}  \documentclass[]{article}